\hsection{Conceptual Modeling / Konzeptuelle Modellierung}%
%
%
\begin{question}{ERDs: Büros / Offices}%
\begin{itemize}%
\item[EN] Design an Entity-Relationship Model (using the original notation with rectangles, diamonds, and ellipses) for this scenario: %
Every office has a room number. %
Every employee has a name, title, and worker's number. %
Every office offers some work stations, each of which having a station number. %
Every employee occupies one of the work stations, starting from a certain date. %
Every office has one or multiple telephones, each with a certain phone number.%
%
\item[DE] Entwerfen Sie ein Entity-Relationship Modell (in der Originalnotation mit Rechtecken, Rauten, und Ellipsen) für folgende Situation: %
Jedes Büro hat eine Zimmernummer. %
Jeder Mitarbeiter hat einen Name, Titel, und Arbeiternummer. %
Jedes Büro bietet eine Menge von Arbeitsplätzen, die jeweils eine Platznummer haben. %
Jeder Mitarbeiter sitzt ab einem gewissen Datum an einem der Arbeitsplätze. %
In jedem Büro gibt es ein or mehrere Telefone mit jeweils einer festen Telefonnummer.%
\end{itemize}%
\end{question}%
%
%
\begin{question}{ERDs: Furniture / Möbel}%
\begin{itemize}%
\item[EN] Design an Entity-Relationship Model (using the original notation with rectangles, diamonds, and ellipses) for this scenario: %
A furniture store sells furniture. %
Each type of furniture has a type-ID, a name, and a price. %
It can be purchased and sold in variants, which are characterized by a name, color, and wood pattern. %
It is produced by a vendor, which is characterized by a vendor-ID, a name, and an address. %
The store has several warehouses, each of which has a name and stores amounts for different variant of different furniture types.
The company has salespeople~(employee-ID, name, date of birth, date of hiring, salary).
A salesperson can acquire and be responsible for arbitrarily many customers~(customer-ID, name, address).
The customers can issue purchase orders, characterized by an order-ID, furniture variant, amount, and deadline.%
%
\item[DE] Entwerfen Sie ein Entity-Relationship Modell (in der Originalnotation mit Rechtecken, Rauten, und Ellipsen) für folgende Situation: %
Ein Möbelladen verkauft Möbel. %
Jeder Möbel-Typ hat eine Typ-ID, einen Namen, und einen Preis. %
Möbel eines Types kann in verschiedenen Varianten gekauft und verkauft werden, welche jeweils einen Namen, eine Farbe und eine Holzmaserung aufweisen. %
Ein Möbeltyp wird von einem Anbieter hergestellt, der eine Anbieter-ID, einen Namen, und eine Adresse hat. %
Jede der Lagerhalle des Möbelladens hat einen Namen und kann Anzahlen verschiedener Varianten verschiedener Möbeltypen speichern. %
Der Laden hat Verkäufer~(Verkäufer-ID, Name, Geburtsdatum, Einstellungsdatum, Gehalt).
Ein Verkäufer kann beliebig viele Kunden~(Kunden-ID, Name, Adresse) anwerben und für diese verantwortlich sein. %
Ein Kunde kann Bestellungen aufgeben, die jeweils eine Bestellungs-ID, Variante eines Möbeltyps, Anzahl, und Deadline haben.%
%
\end{itemize}%
\end{question}%
%
%
\begin{question}{ERDs: Soccer / Fußball}%
\begin{itemize}%
\item[EN] Design an Entity-Relationship Model (using the original notation with rectangles, diamonds, and ellipses) for this scenario: %
We develop a soccer \db. %
Every player has a name and age.
Ever game takes place on a certain game day. %
Every team has a name and a hometown. %
In each game, one team participates as home team and one as guest team. %
We store which player plays in which game, together with the starting and ending minute of their participation. %
Every team has a trainer, of whom we store name and age.%
%
\item[DE] Entwerfen Sie ein Entity-Relationship Modell (in der Originalnotation mit Rechtecken, Rauten, und Ellipsen) für folgende Situation: %
Es wird eine Fußballdatenbank entwickelt. %
Jeder Spieler hat einen Namen und ein Alter. %
Jedes Spiel findet an einem Spieltag statt. %
Jede Mannschaft hat einen Namen und einen Heimatort. %
An jedem Spiel nimmt eine Mannschaft als Heimteam und eine Mannschaft als Gast teil. %
Es wird gespeichert, an welchem Spiel welcher Spieler teilgenommen hat, und zwar jeweils die Anfangs- und die Ende-Minute seines Einsatzes. %
Jede Mannschaft hat einen Trainer, dessen Name und Alter gespeichert werden.%
\end{itemize}%
\end{question}%
%
%
\begin{question}{ERDs: Airline / Fluggesellschaft}%
\begin{itemize}%
\item[EN] Design an Entity-Relationship Model (using the original notation with rectangles, diamonds, and ellipses) for this scenario: %
Every airline has a name, home country, hometown, and abbreviation. %
An airplane has an airplane number and date of last inspection. %
Each airplane belongs to a given type, which is described by its name, number of seats, and top speed. %
Each airplane also belongs to a given airline since a certain date. %
Each pilot has a name, pilot number, qualification level, and flight hours. %
Each pilot works for a given airline since a specific date. %
Passengers are characterized by a passenger number, name, data of birth, and address.
Flights have flight number, date, start airport, destination airport, and flight time.
Each passenger can book an arbitrary number of such flights. %
A flight is realized by a given pilot and airplane.%
%
\item[DE] Entwerfen Sie ein Entity-Relationship Modell (in der Originalnotation mit Rechtecken, Rauten, und Ellipsen) für folgende Situation: %
Jede Fluggesellschaft hat einen Namen, ein Heimatland, eine Heimatstadt, und eine Abkürzung. %
Jedes Flugzeug hat eine Flugzeugnummer und ein Datum der letzten Inspektion. %
Jedes Flugzeug hat einen bestimmten Typ, der durch seinen Name, die Anzahl der Sitzplätze und die Höchstgeschwindikeit charakterisiert wird. %
Jedes Flugzeug gehört zu einer Fluggesellschaft ab einem bestimmten Datum. %
Jeder Pilot hat einen Namen, eine Pilotennummer, eine Qualifikation, und eine Anzahl von Flugstunden. %
Jeder Pilot arbeitet für eine bestimmte Fluggesellschaft ab einem bestimmten Datum. %
Passagiere werden charakterisiert durch eine Passagiernummer, Name, Geburtsdatum und Adresse. %
Flüge haben Flugnummern, Datum, Startflughafen, Zielflughafen, und eine Flugzeit. %
Jeder Passagier kann eine beliebge Anzahl Flüge buchen. %
Ein Flug wird von einem bestimmten Pilot und Flugzeug realisiert.%
\end{itemize}%
\end{question}%
%
%
\begin{question}{ERDs: Company / Firma}%
\begin{itemize}%
\item[EN] Design an Entity-Relationship Model (using the original notation with rectangles, diamonds, and ellipses) for this scenario: %
Every work group has a group number, name, and location. %
Every employee has a worker's number, name, date of birth, salary, address, and job. %
Every employee belongs to a work group. %
Each project has a name and deadline. %
Every employee works on at least one, but maybe multiple projects. %
Each project has an employee as a leader.%
%
\item[DE] Entwerfen Sie ein Entity-Relationship Modell (in der Originalnotation mit Rechtecken, Rauten, und Ellipsen) für folgende Situation: %
Jede Arbeitsgruppe hat eine Gruppennummer, Name, und Ort. %
Jeder Mitarbeiter hat eine Arbeiternummer, Name, Geburtsdatum, Gehalt, Adresse, und Arbeitsaufgabe. %
Jeder Mitarbeiter gehört zu einer Arbeitsgruppe. %
Jedes Projekt hat einen Namen und eine Deadline. %
Jeder Mitarbeiter arbeitet an mindestens einem, vielleicht aber auch mehreren Projekten. %
Jedes Projekt hat einen Mitarbeiter als Leiter.%
\end{itemize}%
\end{question}%
%
%
\begin{question}{ERDs: Sports / Sportarten}%
\begin{itemize}%
\item[EN] Design an Entity-Relationship Model (using the original notation with rectangles, diamonds, and ellipses) for this scenario: %
Every sports discipline has a name and a world, asia-, and olympic record. %
There are competitions which have a competition number, date, location, start time, and type. %
The competition type has a name, e.g., Asian Championship, Olympics, etc. %
Every athlete has a name, age, sports club, and home country. %
Every competition offers a set of sports disciplines and athletes can take part in multiple disciplines in multiple competitions. %
Each athlete has a trainer, for whom we store the qualification, name, and date of birth. %
A trainer can train arbitrarily many athletes.%
%
\item[DE] Entwerfen Sie ein Entity-Relationship Modell (in der Originalnotation mit Rechtecken, Rauten, und Ellipsen) für folgende Situation: %
Jede Sportdisziplin hat einen Namen und jeweils eine Welt-, Asien-, und Olympiarekord. %
Es werden Wettkämpfe ausgetragen, die jeweils eine Wettkampfnummer, Datum, Ort, Uhrzeit und Art haben.
Jede Wettkampfart hat einen Namen, \DEzB\ Asienmeisterschaft, Olympiade, usw. %
Jeder Sportler hat einen Namen, Alter, Sportklub, und Heimatland. %
Jeder Wettkampf kann bestimmte Sportarten anbieten und Sportler können an Wettkämpfen teilnehmen, und zwar jeweils in beliebig vielen Disziplinen. %
Jeder Sportler hat einen Trainer, über den wir den Name, die Qualifikation, und das Geburtsdatum speichern.%
Ein Trainer kann beliebig viele Sportler betreuen.%
\end{itemize}%
\end{question}%
%
%
\begin{question}{ERDs: Hospital / Krankenhaus}%
\begin{itemize}%
\item[EN] Design an Entity-Relationship Model (using the original notation with rectangles, diamonds, and ellipses) for this scenario: %
The work of a hospital should be stored and represented by a database. %
Each patient has a patient-ID, a name, a date of birth, an address, an entry date, and exit date, and a diagnosis. %
Each medicine has a name, a main chemical compound, and a producer. %
Each department has a name and a specialization. %
Each doctor has a doctor-ID, a name, and a phone number. %
Each doctor belongs to a department. %
Each room belongs to a department, has a room number, a number of beds, and a phone number.
Each patient belongs to a room. %
Each patient is treated by one or multiple doctors. %
Each patient gets a specific dosage (number of administrations per day, amount per administration) of one or multiple medicines.%
%
\item[DE] Entwerfen Sie ein Entity-Relationship Modell (in der Originalnotation mit Rechtecken, Rauten, und Ellipsen) für folgende Situation: %
Die Arbeit eines Krankenhauses soll in einer Datenbank repräsentiert werden. %
Jeder Patient hat eine Patienten-ID, einen Namen, ein Geburtsdatum, eine Adresse, ein Einlieferungsdatum, ein Entlassungsdatum, und eine Diagnose. %
Jede Medizin hat einen Namen, einen Hauptwirkstoff, und einen Produzenten. %
Jede Abteilung hat einen Namen und eine Spezialisierung. %
Jeder Arzt hat eine Arzt-ID, einen Namen, und eine Telefonnummer. %
Jeder Arzt gehört zu einer Abteilung. %
Jeder Raum gehört zu einer Abteilung und hat eine Raumnummer, eine Anzahl von Betten, und eine Telefonnummer. %
Jeder Patient gehört zu einem Raum. %
Jeder Patient kann von einem oder mehreren Ärzten behandelt werden. %
Jeder Patient bekommt eine bestimmte Dosierung~(Anzahl der Anwendungen pro Tag, Menge pro Anwendung) von einer oder mehrerer Medizinen.%
\end{itemize}%
\end{question}%
%%
\endhsection%
