\hsectionENDE{Development}{Entwicklung}%
%
%%
\begin{questionENDE}{Models}{Modelle}%
\qENDE{%
The design of databases is often understood as the sequential development of three models~(conceptual model, logical model, and physical model). %
Explain the purpose of each of these models.%
}{%
Das Design von Datenbanken wird oft als Abfolge der Entwicklung dreier Modelle~(konzeptuelles Modell, logisches Modell, und physisches Modell) verstanden. %
Erklären Sie den Zweck jedes dieser Modelle.%
}%
\end{questionENDE}%
%
%%
\begin{questionENDE}{Models}{Modelle}%
\qENDE{%
Explain the difference between the conceptual and the logical schema in \db\ design.%
}{%
Erklären Sie den Unterschied zwischen konzeptuellem und logischem Schema in Datenbank-Design.%
}%
\end{questionENDE}%
%
%%
\begin{questionENDE}{Conceptual Model}{Konzeptuelles Modell}%
\qENDE{%
Why do we often have separate conceptual and logical models in the database design? %
What is the conceptual model good for? %
Why do we not just directly with the logical model?%
}{%
Warum haben wir oft ein separates konzeptuelles und logisches Modell während der Datenbankentwicklung? %
Wofür braucht man ein konzeptuelles Modell? %
Warum fängt man nicht direkt mit dem logischen Modell an?%
}%
\end{questionENDE}%
%
%
\begin{questionENDE}{Requirements}{Anforderungen}%
\qENDE{%
Explain the purpose of and differences between the four types of requirements~(business requirements, functional requirements, non-functional requirements, and constraints).%
}{%
Erklären Sie den Zweck von und die Unterschiede der vier Arten von Anforderungen~(Fachanforderungen, funktionale Anforderungen, nicht-funktionale Anforderungen und Projektgrenzen).
}%
\end{questionENDE}%
%
%
\begin{questionENDE}{Requirements}{Anforderungen}%
\qENDE{%
What is the difference between non-functional requirements and project constraints?%
}{%
Was ist der Unterschied zwischen nicht-funktionalen Anforderungen und den Projektgrenzen?%
}%
\end{questionENDE}%
%
%
\begin{questionENDE}{Requirements}{Anforderungen}%
\qENDE{%
What is the difference between business requirements and functional requirements?%
}{%
Was ist der Unterschied zwischen Fachanforderungen und nicht-funktionalen Anforderungen?%
}%
\end{questionENDE}%
%
%
\begin{questionENDE}{Requirements}{Anforderungen}%
\qENDE{%
Assume that you are supposed to design a \db\ for managing the operations of a restaurant. %
Give three examples for business requirements of such an application.%
}{%
Nehmen Sie an, dass Sie eine Datenbank zum Managen des Betriebs eines Restaurants entwickeln sollen. %
Geben Sie drei Beispiele für Fachanforderungen einer solchen Applikation.%
}%
\end{questionENDE}%
%
%
\begin{questionENDE}{Requirements}{Anforderungen}%
\qENDE{%
Assume that you are supposed to design a \db\ for managing the operations of a naval port. %
Give vier examples for functional requirements of such an application.%
}{%
Nehmen Sie an, dass Sie eine Datenbank zum Managen des Betriebs eines Seehafens entwickeln sollen. %
Geben Sie four Beispiele für funktionalen Anforderungen einer solchen Applikation.%
}%
\end{questionENDE}%
%
%
\begin{questionENDE}{Requirements}{Anforderungen}%
\qENDE{%
Assume that you are supposed to design a \db\ for managing the operations of a history museum. %
Give three examples for non-functional requirements of such an application.%
}{%
Nehmen Sie an, dass Sie eine Datenbank zum Managen des Betriebs eines geschichtlichen Museums entwickeln sollen. %
Geben Sie drei Beispiele für nicht-funktionale Anforderungen einer solchen Applikation.%
}%
\end{questionENDE}%
%
%
\begin{questionENDE}{Requirements}{Anforderungen}%
\qENDE{%
Assume that you are supposed to design a \db\ for managing the operations of a high school. %
Give three examples for project constraints of such an application.%
}{%
Nehmen Sie an, dass Sie eine Datenbank zum Managen des Betriebs einer Schule der Sekundarstufe~II, \DEzB\ eines Gymansiums, entwickeln sollen. %
Geben Sie drei Beispiele für Projektgrenzen einer solchen Applikation.%
}%
\end{questionENDE}%
%
%
\begin{question}{Stakeholders}%
\qENDE{%
A stakeholder is anybody that is concerned with or, in any form, touched by or related to a project, is a user of or has an professional interest in a project. %
Assume that you got the task to develop a database for managing the day-to-day processes in a big hospital with several thousands of patients per day. %
The goal is the complete digitalization of all processes in the hospital. %
Name and discuss six groups of stakeholders in this project. %
For each of them, explain why they are stakeholders and why their opinions and suggestions should be considered during the development process.%
}{%
Als \emph{Stakeholder} wird jede Person bezeichnet, die irgendwie mit einem Projekt in Berührung kommt oder damit in Zusammenhang steht; also jeder, der damit etwas zu tun hat, es verwendet, or einen professionellen Bezug zum Projekt hat. %
Nehmen Sie an, dass Sie die Aufgabe bekommen haben, eine Datenbank zum Managen aller täglicher Prozesse eines großen Krankenhauses mit mehreren tausend Patienten pro Tag zu entwickeln. %
Das Ziel ist die komplette Digitalisierung von allen Projekten des Krankenhauses. %
Nennen und diskutieren Sie sechs Gruppen von Stakeholders in dem Projekt. %
Für jede Gruppe, erklären Sie warum sie Stakeholders sind und warum ihre Meinungen und Vorschläge während des Entwicklungsprozesses berücksichtigt werden sollten.%
}%
\end{question}%
%
%
%
\begin{question}{Stakeholders}%
\qENDE{%
A stakeholder is anybody that is concerned with or, in any form, touched by or related to a project, is a user of or has an professional interest in a project. %
Assume that you got the task to develop a database for managing the day-to-day processes in a big supermarket with several thousands of customers per day. %
The goal is the complete digitalization of all processes in the supermarket. %
Name and discuss five groups of stakeholders in this project. %
For each of them, explain why they are stakeholders and why their opinions and suggestions should be considered during the development process.%
}{%
Als \emph{Stakeholder} wird jede Person bezeichnet, die irgendwie mit einem Projekt in Berührung kommt oder damit in Zusammenhang steht; also jeder, der damit etwas zu tun hat, es verwendet, or einen professionellen Bezug zum Projekt hat. %
Nehmen Sie an, dass Sie die Aufgabe bekommen haben, eine Datenbank zum Managen aller täglicher Prozesse eines großen Supermarkts mit mehreren tausend Kunden pro Tag zu entwickeln. %
Das Ziel ist die komplette Digitalisierung von allen Projekten des Supermarkts. %
Nennen und diskutieren Sie fünf Gruppen von Stakeholders in dem Projekt. %
Für jede Gruppe, erklären Sie warum sie Stakeholders sind und warum ihre Meinungen und Vorschläge während des Entwicklungsprozesses berücksichtigt werden sollten.%
}%
\end{question}%
%
%
%
\begin{questionENDE}{Requirements}{Anforderungen}%
\qENDE{%
Explain four methods that can be used to gather and understand the requirements of a \db\ project.%
}{%
Erklären Sie vier Methoden, die benutzt werden können, um die Anforderungen eines Datenbankprojekts zu sammeln und zu verstehen.%
}%
\end{questionENDE}%
%
\endhsection%
